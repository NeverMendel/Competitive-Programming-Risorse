\documentclass{article}
\usepackage[utf8]{inputenc}

\usepackage[a4paper, total={6.5in, 10in}]{geometry}
\usepackage{setspace}

\usepackage{listings}
\usepackage{xcolor}

\newcommand{\floor}[1]{\lfloor #1 \rfloor}

\lstset{
  language=C++,
  backgroundcolor=\color{black!5}, % set backgroundcolor
  columns=fullflexible,
}

\title{Competitive Programming}
\author{Davide Cazzin}
\date{Novembre 2021}

\onehalfspacing
\begin{document}

\maketitle

\section{Complessità}

Un concetto molto importante nel Competitive Programming è quello della complessità di un algoritmo. Gli algoritmi che scriviamo devono essere efficienti sia in termini di tempo che di spazio. 
Spesso i problemi che dovremo risolvere ci daranno un indicazione sulla dimensione dei dati in input, il tempo massimo per il quale possiamo eseguire il nostro programma e la memoria massima che può utilizzare.

Esistono 3 classi di complessità, tuttavia ora ci concentreremo solo su Big O. Big O viene utilizzato per rappresentare la complessità di un algoritmo nel caso peggiore.

Le classi di complessità possono essere utilizzate per descrivere sia la complessità temporale che la complessità spaziale di un algoritmo.

\section{Esempio complessità}

Vediamo ora un esempio di complessità di alcuni algoritmi.

\lstinputlisting[language=C++, caption=Esempi complessità]{esempiComplessita.cpp}

\subsection{Bubble Sort}

Supponiamo di dover scrivere l'algoritmo Bubble Sort per ordinare gli elementi di un vettore in ordine non decrescente.

\lstinputlisting[language=C++, caption=Bubble Sort in C++]{bubbleSort.cpp}

La compessità temporale di questo algoritmo è $T(n) = n * \frac{n}{2}$, diremo quindi che la classe di complessità è $O(n^2)$, il peggior caso si verifica quando il vettore è ordinato con ordine decrescente. Il miglior caso si verifica quando il vettore è già ordinato in ordine crescente, se tale condizione si verifica il vettore verrà visitato una sola volta e poi l'algoritmo terminerà.

\subsection{Binary Search}

Vediamo la complessità dell'algoritmo Binary Search. Il Binary Search è un algoritmo di ricerca che serve per trovare efficientemente un elemento in un vettore ordinato.

\lstinputlisting[language=C++, caption=Binary Search in C++]{binarySearch.cpp}

Per calcolare esattamente la complessità di questo algoritmo esistono diversi metodi matematici, tuttavia non ci soffermeremo troppo su questo. Nel competitive programming è importante saper calcolare velocemente la complessità.


Ad ogni iterazione del ciclo while, il numero degli elementi che stiamo considerando viene dimezzato. 
Continuiamo il ciclo while fino a che o troviamo l'elemento oppure finiamo gli elementi (se low >= high).
Ad esempio supponiamo che gli elementi siano 7 e non l'elemento che stiamo cercando non sia presente nel vettore.

$ \floor{7/2} = 3 $

$ \floor{3/2} = 1 $

$ \floor{1/2} = 0 $

La funzione matematica che calcola il numero di volte che un numero è divisibile per 2 è la funzione logaritmo, in questo caso il logaritmo in base 2 di 7.

$ log2(7) $

Diciamo quindi che la complessità dell'algoritmo Binary Search è $O(log(n))$.

\section{Sort}

In competitive programming molto difficilmente dovremmo scrivere un algoritmo di sorting, la maggioranza delle volte useremo il sort che ci viene fornito dalla standard library.

\begin{lstlisting}
void sort (RandomAccessIterator first, RandomAccessIterator last);

void sort (RandomAccessIterator first, RandomAccessIterator last, Compare comp);
\end{lstlisting}

Il sort utilizzato da questa funzione è chiamato IntroSort e ha una complessità temporale $O(nlog(n))$ e spaziale $O(log(n))$.

Il comparator di default che viene utilizzato dalla funzione sort, nel caso non ne venga fornito un altro, è $less<int>$ e ordina gli elementi presenti all'interno della struttura dati in ordine non decrescente. Per ordinare il vettore con ordine non crescente possiamo usare il comparator $greater<int>$, come nel seguente esempio:

\begin{lstlisting}
vector<int> v = {1,4,5,6,3,2};
sort(v.begin(), v.end(), greater<int>);
\end{lstlisting}

Nel caso volessimo ordinare il nostro vettore secondo un criterio diverso da quello del comparator $less<int>$ e $greater<int>$ allora dobbiamo definire un comparator. Supponiamo ad esempio di aver un vettore di $pair<string, int>$ e vogliamo che il vettore sia ordinato in base all'intero con ordine non crescente e nel caso di interi uguali devono essere ordinati in ordine alfabetico.

\lstinputlisting[language=C++, caption=Custom Comparator]{sortPairs.cpp}

\section{Makefile}

\lstinputlisting[caption=Makefile example]{Makefile}

\section{Template}

\lstinputlisting[caption=Template C++]{template.cpp}

\end{document}
