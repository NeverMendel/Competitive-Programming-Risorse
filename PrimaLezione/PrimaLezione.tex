\documentclass{article}
\usepackage[utf8]{inputenc}

\usepackage[a4paper, total={6.5in, 10in}]{geometry}
\usepackage{setspace}

\usepackage{listings}
\usepackage{xcolor}

\lstset{
  language=C++,
  backgroundcolor=\color{black!5}, % set backgroundcolor
  columns=fullflexible,
}

\title{Competitive Programming}
\author{Davide Cazzin}
\date{Novembre 2021}

\onehalfspacing
\begin{document}

\maketitle

\section{Complessità}

Un concetto molto importante nel Competitive Programming è quello della complessità di un algoritmo. Gli algoritmi che scriviamo devono essere efficienti sia in termini di tempo che di spazio. 
Spesso i problemi che dovremo risolvere ci daranno un indicazione sulla dimensione dei dati in input, il tempo massimo per il quale possiamo eseguire il nostro programma e la memoria massima che può utilizzare.

Vediamo 3 classi di complessità, big $O$, big theta ($\Theta$) e big omega ($\Omega$).
\begin{itemize}
  \item Big O: Complessità asintotica, complessità dell'algoritmo nel peggior caso possibile. Definisce un limite superiore sulla complessità dell'algoritmo.
  \item Big theta: Se big O e big omega sono uguali allora si può dire che big theta è uguale a big O e big omega.
  \item Big omega: Complessità dell'algoritmo nel miglior caso possibile. Definisce un limite inferiore sulla complessità dell'algoritmo.
\end{itemize}

Le classi di complessità possono essere utilizzate per descrivere sia la complessità temporale che la complessità spaziale di un algoritmo.

\section{Esempio complessità}

Vediamo ora un esempio di complessità di un algoritmo. Supponiamo di dover scrivere l'algoritmo Bubble Sort per ordinare gli elementi di un vettore in ordine non decrescente.

\lstinputlisting[language=C++, caption=Bubble Sort in C++]{bubbleSort.cpp}

La compessità temporale di questo algoritmo è $T(n) = n * \frac{n}{2}$, diremo quindi che la classe di complessità è $O(n^2)$, il peggior caso si verifica quando il vettore è ordinato con ordine decrescente. Il miglior caso si verifica quando il vettore è già ordinato in ordine crescente, se tale condizione si verifica il vettore verrà visitato una sola volta e poi l'algoritmo terminerà, possiamo quindi dire che $\Omega(n)$.

\section{Sort}

In competitive programming molto difficilmente dovremmo scrivere un algoritmo di sorting, la maggioranza delle volte useremo il sort che ci viene fornito dalla standard library.

\begin{lstlisting}
void sort (RandomAccessIterator first, RandomAccessIterator last);

void sort (RandomAccessIterator first, RandomAccessIterator last, Compare comp);
\end{lstlisting}

Il sort utilizzato da questa funzione è chiamato IntroSort e ha una complessità temporale $O(nlogn)$ e spaziale $O(logn)$.

Il comparator di default che viene utilizzato dalla funzione sort, nel caso non ne venga fornito un altro, è $less<int>$ e ordina gli elementi presenti all'interno della struttura dati in ordine non decrescente. Per ordinare il vettore con ordine non crescente possiamo usare il comparator $greater<int>$, come nel seguente esempio:

\begin{lstlisting}
vector<int> v = {1,4,5,6,3,2};
sort(v.begin(), v.end(), greater<int>);
\end{lstlisting}

Nel caso volessimo ordinare il nostro vettore secondo un criterio diverso da quello del comparator $less<int>$ e $greater<int>$ allora dobbiamo definire un comparator. Supponiamo ad esempio di aver un vettore di $pair<string, int>$ e vogliamo che il vettore sia ordinato in base all'intero con ordine non decrescente e nel caso di interi uguali devono essere ordinati in ordine alfabetico.

\lstinputlisting[language=C++, caption=Custom Comparator]{sortPairs.cpp}

\section{Makefile}

\lstinputlisting[caption=Makefile example]{Makefile}

\end{document}
